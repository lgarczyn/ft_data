%******************************************************************************%
%                                                                              %
%                  ft_data.en.tex for LaTeX                                     %
%                  Created on : Tue Mar 10 13:27:28 2015                       %
%                  Made by : David "Thor" GIRON <thor@42.fr>                   %
%                  Updated by : Catherine "Cat" MADINIER <cat@42.fr>           %
%                                                                              %
%******************************************************************************%

\documentclass{42-en}
\graphicspath{ {./images/} }

%******************************************************************************%
%                                                                              %
%                                    Header                                    %
%                                                                              %
%******************************************************************************%
\begin{document}



                           \title{ft_data}
                          \subtitle{Now go, and cache miss no more}
                       \member{Louis Garczynski}{lgarczyn@student.42.fr}
                        \member{42 Staff}{pedago@42.fr}

\summary {
  An introduction from simple to mid-level performance-oriented generic data structures
}

\maketitle

\tableofcontents


%******************************************************************************%
%                                                                              %
%                                  Foreword                                    %
%                                                                              %
%******************************************************************************%
\chapter{Foreword}

    \includegraphics{universe}

    Your name is KANAYA MARYAM.\\

    You are one of the few of your kind who can withstand the BLISTERING ALTERNIAN SUN,
    and perhaps the only who enjoys the feel of its rays.
    As such, you are one of the few of your kind who has taken a shining to LANDSCAPING.
    You have cultivated a lush oasis around your hive, and in particular,
    you have honed your craft through the art of TOPIARY,
    sculpting your trees to match the PUFFY ORACLES from your dreams.
    You have embraced the tool of this trade,
    which conveniently is the weapon of choice for those who would hunt the
    HEINOUS BROODS OF THE UNDEAD which crawl from the sand at sunrise to feast on the light and the living.\\

    It would be convenient if you actually hunted them,
    but it is of course far too dangerous,
    every bit as suicidal as attempting to poach the terrible MUSCLEBEASTS who roam at night.
    So you indulge in your bright fascination with the grim through literature.
    Just before the sun goes down and you join your flora in rest,
    you immerse yourself in tales of RAINBOW DRINKERS and SHADOW DROPPERS and FORBIDDEN PASSION.\\

    You are one of the few of your kind with JADE GREEN BLOOD.
    As such you are one of the few who could be selected and raised by a VIRGIN MOTHER GRUB,
    an event so rare as to elude documented precedent.
    She would defend you from desert threats,
    and though her life would be short,
    in time you would assure her of progeny.\\

    You are one of the few of your kind whose affection for the aesthetic strongly overpowers
    instinctive regard for the utilitarian.
    As such, you are one of the few of your kind who has developed a zeal for FASHION and DESIGN and LIVELY COLORFUL PATTERNS.
    You decorate your hive with FLORA and FABRIC, as delicately or aggressively as inspiration demands.
    You are a SEAMSTRESS or a RAGRIPPER or a TREETRIMMER or a LUMBERJACK, whichever you care to be,
    and your unique hive is equipped with a great supply of advanced technology to accommodate your interests.
    The technology and indeed the hive itself were all recovered from the ruins nearby when you were very young.
    The seed of your hive was deployed on the volcanic rocks beneath the sand with the assistance of your lusus
    and her remarkable burrowing skills, and you have lived there happily together since.\\

    You know the ruins and the hive and everything here that is not sand and rock originated from the world of your dreams.
    You also know that one day you will visit this world while you are awake. That day is today.\\

    Your trolltag is grimAuxiliatrix and you Tend To Enunciate Each Word You Speak Very Clearly And Carefully\\

    What will you do?\\

    So, let's use the foreword section of this \texttt{42} subject
    template to introduce the content of this document and
    its goals.
    Specifically, the formatting of a trivial
    \texttt{LaTeX} document and the normalized chaptering of our
    subjects. If you read this from the pdf, don't forget to open the
    source file (\texttt{sample.en.tex} file) next to it, in
    order to see behind the scenes and to understand what command
    generates what result. Otherwise, if you have started with the
    sources, congrats, that's the spirit! But open the pdf
    (\texttt{sample.en.pdf} file) anyway to double check.\\

    What can you do if the \texttt{sample.en.pdf} file is not available?
    Easy, just compile the source file \texttt{sample.en.tex} using
    the shell command \texttt{make}. Please refer to the documentation
    to set up \texttt{LaTeX} on your system if needed.\\

    If you're not familiar with \texttt{LaTeX}'s syntax, here is a
    fairly comprehensive list of everything you'll need to write your
    subject.\\


    \section{Example of section}


        \subsection{Example of sub-section}

           This sub-section is empty.


        \newpage


        \subsection{A bullet point list}

            \begin{itemize}\itemsep1pt
                \item what
                \item a
                \item wonderful
                \item list.\\
            \end{itemize}


        \subsection{A descriptions list}

            \begin{description}\itemsep3pt
                \item [Orange:] Round and orange fruit.
                \item [Strawberry:] Strawberry shaped fruit. Also red.
                \item [Cucumber:] Phallus-shaped green vegetable.\\
            \end{description}


        \subsection{An enumeration}

            An enumeration of the reasons why I like you:\\

            \begin{enumerate}\itemsep7pt
                \item You are smart.
                \item Your are very talented.
                \item Your are magnificent.
                \item I'm a nice person.
            \end{enumerate}


        \subsection{Urls and links}

            If you have no clue how to insert links or urls in your
            document, search for an online explanation using
            \href{www.google.com}{Google}. Please note that \texttt{Google}
            is available at \url{www.google.com}.


        \newpage


        \subsection{An info box}

            \info{
              For information, please read this info box.
            }


        \subsection{A hint box}

            \hint {
              You should read this hint box, really.
            }


        \subsection{A warning box}

            \warn {
              Beware! This is a warning box!
            }


        \newpage


        \subsection{A \texttt{shell} snippet}

           \begin{42console}
$sudo rm -rf /\end{42console}



        \subsection{A \texttt{C} code snippet}

           \begin{42ccode}
int main( void ) {

    puts( "hello world !" );
    return 0;
}
\end{42ccode}


        \subsection{A \texttt{C++} code snippet}

            \begin{42cppcode}
int main( void ) {

    std::cout << "hello world !" << std::endl;
    return 0;
}
\end{42cppcode}


        \subsection{A picture !}

            \begin{figure}[H]
                \begin{center}
                    \includegraphics[width=8cm]{42.png}
                \end{center}
            \end{figure}


        \newpage


        \subsection{Some special characters}

            \begin{description}\itemsep1pt
                \item [Underscore:] \_
                \item [Ampersand:] \&
                \item [Dollar:] \$
                \item [Elipsis:] \dots
            \end{description}


    \section{About chaptering}

    Each chapter of the pdf must appear in your subject,
    \textbf{including} the \texttt{Foreword} chapter. For your
    convenience, the best way to use this template \texttt{LaTeX} file is
    to copy it and rename it, then replace the provided descriptions by
    your own content.\\

    \warn{
      If you are part of a company, the \texttt{Foreword} chapter is
      the best place to write about your business, the context
      of this project, introduce yourself and/or your team, etc.
    }

%******************************************************************************%
%                                                                              %
%                                 Introduction                                 %
%                                                                              %
%******************************************************************************%
\chapter{Introduction}

    So you've learned how to make linked lists, and binary trees. You're proud of yourself.\\

    Don't be.\\

    In fact, accept that both of these are terrible. Any data structure that calls
    malloc at every insert is terrible. No exceptions\\

    If you are familiar with the foreword, it meant you might have read the bestseller "Data Structures for Assholes".
    I could have used that as a foreword, but hey, it might have relevant, and we don't want that.

    Your first data structure will the be the dynamic array. Excidingly boring, but you won't look
    inadequate in front of your future spouse's family when they ask "how do you store text" and you reply
    "character by character in a linked list" like a moron.

    Second will be the bitset. Because who can say no to reducing your memory overhead by 8.

    Third will be a queue, because you gotta learn to write that at some point.

    Fourth will be a sorted array, to get yourself ready for the

    Fifth, a packed memory array, or the poor-man's-btree, and the smart-man's-hashtable.

    Google each of them, ask yourself "can I do that", then think again.


%******************************************************************************%
%                                                                              %
%                                  Goals                                       %
%                                                                              %
%******************************************************************************%
\chapter{Goals}

    To not use linked lists. To delete them from your libft. To never. Speak. Of. Them. Again.

    But seriously, the goal is not just an introduction to data structures, but also a tool
    that can be used with any 42 C projects and drastically reduce teh amount of copy-pasted code, such
    as sorts, reallocs, inserts, etc.

    If you do your job well, you'll have a safe library of extremely multi-purpose tools for any situation.

    I even included a 3000-lines colorful testing main ! How kind of me.


%******************************************************************************%
%                                                                              %
%                             General instructions                             %
%                                                                              %
%******************************************************************************%
\chapter{General instructions}

    This is a C, Norme compliant library project.

    As such, your makefile should produce a "libdata.a" file. It should ideally be placed in a "data"
    folder, in your root or libft folder. Another Makefile will link the static library/libraries and compile them
    with the main.c provided in the resources.

    Your library folder must contain a "data.h" file based on the one provided in resources.

    The whole program must compile without warnings or errors.

    Both Makefiles must only compile what is necessary, and be able to fully clean the object files left behind,
    as for any 42 projects.

    The maximum allowed optimisation level is -O2

    Your program must not crash or cause leaks during valid use. Valid use is defined as:

    "anything in the main.c file, and anything that a user might reasonably do"

    As such, should the main file show any leaks, or crash anytime, you will get 0.

    Correctors will be encouraged to write their own tests.

    Crashes, leaks and memory corruptions are allowed only if the alternative would come at
    an unreasonable cost is performance, API simplicity or freedom in use.

    Should you hesitate between safe and performant, usually choose safe.

    Should you wish to change the main file, for a SIMPLE and REASONABLE change,
    that DOESN'T give you an unfair advantage on ANY part of the project, you main join a diff
    file to be applied to the main during correction.

    You may add personnal tests to the repository, but they must be norme-compliant. 

    You will only have access to these functions, as defined in the main file,

        * xmalloc
        * xfree
    
    As well as

        * write

    but only for error messages and display functions. 

    You are ecouraged to use

        * malloc_usable_size
        * printf

    during development.

    You must *sigh* have an auteur file in your root. It can however contain anything you want.




%******************************************************************************%
%                                                                              %
%                             Mandatory part                                   %
%                                                                              %
%******************************************************************************%
\chapter{Mandatory part}

    Fully implement the API, as used by the main.c with the flags 

    * TEST_ARRAY
    * TEST_BITSET
    * TEST_QUEUE
    * TEST_SORTED
    * TEST_PMA

    turned on. 

%******************************************************************************%
%                                                                              %
%                                 Bonus part                                   %
%                                                                              %
%******************************************************************************%
\chapter{Bonus part}

    Fully implement 5 of the 6 bonus options from the main.c file:

    * TEST_ARRAY_BONUS
    * TEST_BITSET_BONUS
    * TEST_SORTED_BONUS
    * TEST_PMA_BONUS_IT
    * TEST_PMA_BONUS_IT_BACK
    * TEST_PMA_BONUS_MULTI

    Provide performances at least half as fast as the provided baseline.

    Should you have both performances and coverage, you may try for 5 more bonus points.

    Suggested bonuses includes:

    * a define allowing the entire library to switch between the real "realloc" and your
    implementation. You are not allowed to use it to boost your performances score though.

    * an iterator for the other data structures

    * additional functions such as array_concat, or pma_filter

    Use stdlib, boost and rust APIs for inspiration!


%******************************************************************************%
%                                                                              %
%                           Turn-in and peer-evaluation                        %
%                                                                              %
%******************************************************************************%
\chapter{Turn-in and peer-evaluation}

    This part describes the conditions and instructions for the turn-in and
    peer-evaluation of the project. If your project does not
    require any specific turn-in or peer-evaluation instruction, feel free to
    use the following paragraph as is:\\

    Turn in your work using your \texttt{GiT} repository, as
    usual. Only the work that's in your repository will be graded during
    the evaluation.



%******************************************************************************%
\end{document}
