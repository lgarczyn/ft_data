%******************************************************************************%
%                                                                              %
%                  sample.fr.tex for LaTeX                                     %
%                  Created on : Tue Mar 10 13:27:28 2015                       %
%                  Made by : David "Thor" GIRON <thor@42.fr>                   %
%                  Updated by : Catherine "Cat" MADINIER <cat@42.fr>           %
%                                                                              %
%******************************************************************************%

\documentclass{42-fr}


%******************************************************************************%
%                                                                              %
%                                  Prologue                                    %
%                                                                              %
%******************************************************************************%
\begin{document}



                           \title{Document d'exemple}
                          \subtitle{Sous-titre de fou}
                         \member{Ton nom}{ton@mail.com}
                      \member{42 Staff}{pedago@staff.42.fr}

\summary {
  Ce document est un exemple d'utilisation de LaTeX et de la
  classe de style maison de \href{www.42.fr}{42}.
}

\maketitle

\tableofcontents


%******************************************************************************%
%                                                                              %
%                                  Préambule                                   %
%                                                                              %
%******************************************************************************%
\chapter{Préambule}

    Le pr\'eambule d'un sujet \texttt{42} est g\'en\'eralement sans
    rapport avec le sujet en lui-m\^eme. Il s'agit ici de partager
    un trait d'humour (souvent discutable) ou une curiosit\'e avec la
    communaut\'e \texttt{42}.\\





    % Les sauts de lignes dans le code source n'ont pas d'influence
    % sur le pdf généré. Les lignes vides au-dessus et au-dessous
    % n'apparaitront pas. A la place, utilisez \newline (ou son
    % raccourci \\) et \newpage pour créer des sauts de ligne ou des
    % sauts de page.





    Nous profiterons donc du pr\'eambule de ce sujet \texttt{42}
    d'exemple pour introduire son contenu et son int\'er\^et, en
    particulier la structure d'un document LaTeX simple et le
    chapitrage normalis\'e de nos sujets. Si vous lisez ces lignes
    depuis le pdf, pensez \`a ouvrir les sources (fichier
    \texttt{sample.fr.tex}) \`a cot\'e du pdf pour voir l'envers du
    d\'ecor et comprendre quelles commandes g\'en\`erent quels
    r\'esultats dans le pdf. Si au contraire, vous avez commenc\'e par
    ouvrir les sources, f\'elicitations, vous avez l'esprit. Mais
    ouvrez quand m\^eme le pdf (fichier \texttt{sample.fr.pdf}) pour
    garder le fil.\\

    Que faire si le fichier \texttt{sample.fr.pdf} n'est pas pr\'esent ?
    Facile, il suffit de compiler les sources
    (fichier \texttt{sample.fr.tex}) en utilisant la commande
    \texttt{make}. Veuillez vous r\'ef\'erer \`a la documentation
    pour installer LaTeX sur votre syst\`eme si n\'ecessaire.\\

    Si vous n'\^etes pas familliers avec la syntaxe de LaTeX, voici
    une liste assez exhaustive de ce dont vous aurez besoin pour
    r\'ediger votre sujet :\\

    \section{Example de section}


        \subsection{Example de sous-section}

            Cette sous-section est vide.


        \newpage


        \subsection{Liste \`a puces}

            \begin{itemize}\itemsep1pt
                \item what
                \item a
                \item wonderful
                \item list.\\
            \end{itemize}


        \subsection{Liste de descriptions}

            \begin{description}\itemsep3pt
                \item [Orange :] Fruit rond et orange.
                \item [Fraise :] Fruit en forme de fraise, et rouge aussi.
                \item [Concombre :] L\'egume vert de forme phallique.\\
            \end{description}


        \subsection{Une \'enum\'eration}

            Une \'enum\'eration des raisons pour lesquelles je vous
            aime :\\

            \begin{enumerate}\itemsep7pt
                \item Vous \^etes intelligent(e).
                \item Vous \^etes tr\`es talentueux(se).
                \item Vous \^etes magnifique.
                \item Je suis quelqu'un de gentil.
            \end{enumerate}


        \subsection{Urls et liens}

            Si vous n'avez aucune id\'ee de comment ins\'erer des
            liens ou des urls dans votre document, cherchez une
            explication en ligne en utilisant
            \href{www.google.com}{Google}. Veuillez noter que
            \texttt{Google} est disponible \`a l'adresse
            \url{www.google.com}.


        \newpage


        \subsection{Une bo\^ite d'info}

            \info{
              Pour information, merci de lire cette bo\^ite d'information.
            }


        \subsection{Une bo\^ite de conseil}

            \hint {
              Vous devriez lire cette bo\^ite de conseil, vraiment.
            }


        \subsection{Une bo\^ite d'avertissement}

            \warn {
              Prenez garde ! Ceci est une bo\^ite d'avertissement !
            }


        \newpage


        \subsection{Une image !}

            \begin{figure}[H]
                \begin{center}
                    \includegraphics[width=8cm]{42.png}
                \end{center}
            \end{figure}


        \newpage




     \section{Concernant le chapitrage}

     Chaque chapitre de ce pdf doit \^etre pr\'esent dans votre sujet,
     y compris le chapitre \texttt{Pr\'eambule}. Pour votre confort,
     la meilleure facon d'utiliser ce fichier LaTeX d'exemple est de
     simplement le copier et de le renommer, puis de remplacer les
     descriptions fournies dans chaque chapitre par votre propre
     contenu.\\

     \warn{
       Si vous faites parti d'une entreprise, le chapitre
       \texttt{Pr\'eambule} est l'endroit le plus adapt\'e pour
       \'ecrire au sujet de votre entreprise, du contexte de ce
       projet, vous pr\'esenter vous et/ou votre \'equipe, etc.
     }


%******************************************************************************%
%                                                                              %
%                                 Introduction                                 %
%                                                                              %
%******************************************************************************%
\chapter{Introduction}

    L'introduction est une pr\'esentation des grandes lignes du
    projet. Il est appr\'eci\'e de donner un peu de contexte et une
    id\'ee du travail \`a r\'ealiser. Ainsi en lisant ces quelques
    lignes, un \'etudiant peut avoir une vue d'ensemble des th\`emes
    abord\'es.



%******************************************************************************%
%                                                                              %
%                                  Objectifs                                   %
%                                                                              %
%******************************************************************************%
\chapter{Objectifs}

    Il s'agit ici d'expliquer l'int\'er\^et p\'edagogique du projet,
    car au-del\`a de sa forme, un projet est avant tout un
    pr\'etexte \`a la d\'ecouverte ou \`a l'approfondissement d'une
    notion ou d'un groupe de notions. Par exemple, \texttt{42} propose
    un projet de \texttt{C++} appel\'e \texttt{Nibbler}. Sous
    l'apparence d'un simple jeu de \texttt{Snake}, ce projet permet
    d'initier les \'etudiants \`a la cr\'eation d'API et de plugins
    pour un programme en \texttt{C++}.



%******************************************************************************%
%                                                                              %
%                             Consignes générales                              %
%                                                                              %
%******************************************************************************%
\chapter{Consignes g\'en\'erales}

    Cette section regroupe les consignes de base d'un
    projet. Langages, restrictions, autorisations, compilation, etc.



%******************************************************************************%
%                                                                              %
%                             Partie obligatoire                               %
%                                                                              %
%******************************************************************************%
\chapter{Partie obligatoire}

    Partie centrale d'un sujet, la partie obligatoire d\'ecrit en
    d\'etail le travail \`a r\'ealiser et les possibles outils et/ou
    technologies impose\'es. Tout le secret d'un bon sujet
    r\'eside dans l'\'equilibre subtile entre \^etre pr\'ecis et
    laisser une part \`a l'interpr\'etation et \`a l'imagination. Ce
    facteur est important puisque c'est le moteur des discussions et
    des confrontations lors des soutenances de peer-\'evaluation.



%******************************************************************************%
%                                                                              %
%                                 Partie bonus                                 %
%                                                                              %
%******************************************************************************%
\chapter{Partie bonus}

    Lorsqu'on a investi du temps sur un projet et que le r\'esultat
    est au rendez-vous, il est naturel d'avoir envie d'aller plus loin !
    La section bonus propose des ouvertures pour r\'epondre \`a
    cette envie. Bien entendu, la partie bonus n'est accessible que si
    et seulement si la partie obligatoire a \'et\'e r\'ealis\'ee
    enti\`erement et parfaitement.


%******************************************************************************%
%                                                                              %
%                           Rendu et peer-évaluation                           %
%                                                                              %
%******************************************************************************%
\chapter{Rendu et peer-\'evaluation}

    Cette section d\'ecrit les conditions et les instructions
    concernant le rendu et la peer-\'evaluation du projet. Si votre
    projet ne requiert pas de consignes de rendu ou de peer-evaluation
    sortant de l'ordinaire, vous pouvez utiliser le paragraphe suivant
    en l\'etat:\\

    Rendez-votre travail sur votre d\'epot \texttt{GiT} comme
    d'habitude. Seul le travail pr\'esent sur votre d\'epot sera
    \'evalu\'e en soutenance.



%******************************************************************************%
\end{document}
