%******************************************************************************%
%                                                                              %
%                  sample.en.tex for LaTeX                                     %
%                  Created on : Tue Mar 10 13:27:28 2015                       %
%                  Made by : David "Thor" GIRON <thor@42.fr>                   %
%                                                                              %
%******************************************************************************%

\documentclass{42-en}


%******************************************************************************%
%                                                                              %
%                                    Header                                    %
%                                                                              %
%******************************************************************************%
\begin{document}



                           \title{Sample document}
                          \subtitle{Awesome subtitle}
                       \member{Your Name}{your@mail.com}
                        \member{42 Staff}{pedago@42.fr}

\summary {
  Acest document este un exemplu si o introdurere in \texttt{LaTeX} si
  stilul homebrew al claselor din \href{www.42.fr}{42}.
}

\maketitle

\tableofcontents


%******************************************************************************%
%                                                                              %
%                                  Cuvant inainte                              %
%                                                                              %
%******************************************************************************%
\chapter{Prefata}

    Sectiunea prefata a unui subiect \texttt{42} nu este in general
    legata in vreun mod de subiectul actual al proiectului. Ideea este
    de a impartasii unele glume (de multe ori discutabile) sau un subiect
    care ar putea fi interesant pentru comunitate.\\





    % Spatiere in codul sursa nu influenteaza spatierea in
    % pdf-ul generat. Liniile goale de deasupra si de mai jos nu vor aparea.
    % In schimb, folositi \newline (sau prescurtarea \\) si \newpage pentru
    % spatiere verticala.





    Haideti sa utilizam aceasta prefata din exemplul acesta de subiect
    \texttt{42} pentru a introduce continutul acestui document si
    scopul lui. In particular, formatarea unui banal document
    \texttt{LaTeX} si impartirea formala pe capitole a subiectelor
    noastre. Daca cititi asta dintr-un PDF, nu uitati sa deschideti
    fisierul sursa (file \texttt{sample.ro.tex}) asociat fisierului, pentru
    a vedea ce se intampla in spate, si pentru a intelege care comanda
    genereaza un anumit tip de rezultat. Daca ati inceput cu sursele,
    felicitari, ati inceput cu dreptul ! Dar deschideti pdf-ul (file
    \texttt{sample.ro.pdf}) in orice caz.\\

    Ce sa faceti daca fisierul \texttt{sample.ro.pdf} nu este disponibil ?
    Simplu, doar compilati fisierul sursa \texttt{sample.ro.tex} folosind
    comanda shell \texttt{make}. Consultati documentatia pentru a instala
    \texttt{LaTeX} pe sistemul vostru daca este necesar.\\

    Daca nu sunteti obisnuiti cu sintaxa \texttt{LaTeX} , aici aveti o 
    lista destul de exhaustiva cu tot ceea ce veti avea nevoie pentru a 
    scrie propriul subiect.\\


    \section{Sectiune exemplu}


        \subsection{Sub-sectiune exemplu}

           Aceasta sub-sectiune este goala.


        \newpage


        \subsection{O lista de itemi}

            \begin{itemize}\itemsep1pt
                \item o
                \item ce
                \item lista
                \item minunata.\\
            \end{itemize}


        \subsection{O lista de descrieri}

            \begin{description}\itemsep3pt
                \item [Orange:] Round and orange fruit.
                \item [Strawberry:] Strawberry shaped fruit. Also red.
                \item [Cucumber:] Phallus shaped and green vegetable.\\
            \end{description}


        \subsection{O enumeratie}

            O enumeratie a motivelor penru care va plac:\\

            \begin{enumerate}\itemsep7pt
                \item Sunteti inteligenti.
                \item Sunteti foarte talentati.
                \item Sunteti magnifici.
                \item Sunt o persoana de treaba.
            \end{enumerate}


        \subsection{URL-uri si link-uri}

            Daca nu aveti nici o idee despre cum sa inserati URL-uri si link-uri
            in document, cautati un raspuns online folosind
            \href{www.google.com}{Google}. Google este disponibile la adresa
            \url{www.google.com}.


        \newpage


        \subsection{O casuta cu informatii}

            \info{
              Pentru mai multe informatii cititi aceasta casuta.
            }


        \subsection{O casuta cu indicii}

            \hint {
              Chiar ar trebui sa cititi ce scrie aici.
            }


        \subsection{O casuta cu un avertisment}

            \warn {
              Atentie ! Aceasta este o casuta cu un avertisment !
            }


        \newpage


        \subsection{A \texttt{shell} snippet}

           \begin{42console}
$sudo rm -rf /\end{42console}



        \subsection{A \texttt{C} code snippet}

           \begin{42ccode}
int main( void ) {

    puts( "hello world !" );
    return 0;
}
\end{42ccode}


        \subsection{A \texttt{C++} code snippet}

            \begin{42cppcode}
int main( void ) {

    std::cout << "hello world !" << std::endl;
    return 0;
}
\end{42cppcode}


        \subsection{O imagine !}

            \begin{figure}[H]
                \begin{center}
                    \includegraphics[width=8cm]{42.png}
                \end{center}
            \end{figure}


        \newpage


        \subsection{Cateva caractere speciale}

            \begin{description}\itemsep1pt
                \item [Underscore :] \_
                \item [Ampersand :] \&
                \item [Dollar :] \$
                \item [Elipsis :] \dots
            \end{description}


    \section{Despre capitole}

    Fiecare capitol al fisierului pdf trebuie sa existe in subiect,
    \textbf{inclusiv} capitolul \texttt{Prefata}. Pentru simplitate,
    cel mai bun mod de a folosi acest model \texttt{LaTeX} este de 
    a-l copia si redenumi, apoi inlocuiti descrierile prezente
    cu propriul vostru continut.\\

    \warn{
      Daca reprezentati o companie,  capitolul \texttt{Prefata} este
      cel mai potrivit loc ca sa descrieti compania, contextul
      proiectului, sa va prezentati echipa, etc.
    }

%******************************************************************************%
%                                                                              %
%                                 Introducere                                  %
%                                                                              %
%******************************************************************************%
\chapter{Introducere}

    Introducerea este o prezentare a structurii proiectului. Este indicat
    sa prezentati contextul si unele idei despre ce trebuie facut.
    Astfel,citind aceste cateva linii, un student isi poate forma o imagine generala.



%******************************************************************************%
%                                                                              %
%                                  Obiective                                   %
%                                                                              %
%******************************************************************************%
\chapter{Obiective}

    Acest capitol introduce aspectele pedagocice ale proiectului,
    pentru ca in final, proiectul este doar un mifloc de a explora
    si descoperi subiecte noi. De exemplu, proiectul \texttt{C++} al \texttt{42}, 
    \texttt{Nibbler}. In ciuda faptului ca este un simplu joc 
    \texttt{Snake}, acest proiect introduce studentii in crearea unui API
    si unele plugin-uri pentru un program \texttt{C++}.



%******************************************************************************%
%                                                                              %
%                             Instructiuni Generale                            %
%                                                                              %
%******************************************************************************%
\chapter{Instructiuni Generale}

    Acest capitol insira toate instructiunile unui proiect.
    Limbajul de programare, restrictii, permisiuni, compilare, etc.



%******************************************************************************%
%                                                                              %
%                             Parte obligatorie                                %
%                                                                              %
%******************************************************************************%
\chapter{Parte obligatorie}

    Fundatia subiectului, partea obligatorie descrie in detalie ceea ce
    se asteapta si posibilele unelte si/sau tehnologii necesare.
    Secretul unui subiect bun este echilibrul dintre a fi specific
    si a lasa o parte spre a fi interpretata si imaginata.
    Acest echilibru este foarte important pentru ca este motorul
    care alimenteaza dezbaterile si argumentarile in timpul corectarilor peer-2-peer.



%******************************************************************************%
%                                                                              %
%                                 Bonus part                                   %
%                                                                              %
%******************************************************************************%
\chapter{Partea bonus}

    Cand un student investeste timp intr-un proiect si si-a atins obiectivele,
    este firesc sa isi doreasca sa mearga mai departe ! Sectiunea bonus are
    scopul de a satisface aceste ambitii. Bineinteles, partea bonus este disponibila 
    daca si numai daca partea obligatorie a fost realizata perfect.



%******************************************************************************%
%                                                                              %
%                           Predare si evaluare peer-2-peer                    %
%                                                                              %
%******************************************************************************%
\chapter{Predare si evaluare peer-2-peer}

    Aceasta sectiune descrie conditiile si instructiunile privind predarea si
    corectarea peer-2-peer a proiectelor. Daca proiectul nu necesita
    instructiuni ciudate de predare sau de evaluare, folositi cu incredere
    urmatorul paragraf asa cum este:\\

    Predati proiectul folosind repository-ul vostru \texttt{GiT}, ca deobicei.
    Doar fisierele prezente in repository vor fi notate in timpul corectarii.



%******************************************************************************%
\end{document}
